\documentclass{article}
\usepackage[framed,numbered,autolinebreaks,useliterate]{mcode}
\usepackage[utf8]{inputenc}
\usepackage[T1]{fontenc}
\usepackage[polish]{babel}
\usepackage{lmodern}
\selectlanguage{polish}
\usepackage{float}
\usepackage{hyperref}
\usepackage{color}
\usepackage{graphicx} 
\usepackage{float}
\hypersetup{
    colorlinks,
    citecolor=black,
    filecolor=black,
    linkcolor=black,
    urlcolor=black
}
\author{Bartosz Rajkowski}
\title{Projekt I STP\\zadanie 1.9}
\date{30 kwietnia 2017}
\begin{document}
\maketitle
\newpage
\tableofcontents
\newpage
\section*{Dane}
$$
G(s)=\frac{0,5s^2+3,5s+5,625}{s^3+8s^2-36s-288}
$$
\section{Zadanie 1}
\subsection{Treść}

Wyznaczyć transmitancję dyskretną G(z), stosując ekstrapolator zerowego rzędu i przyjmując okres próbkowania Tp=0,1 s. Określić zera i bieguny obydwu transmitancji. Odpowiedzieć na pytanie, czy obiekt jest stabilny.

\subsection{Program}
\lstinputlisting[caption={zad1.m}, label={lst:zad1}]{../zad1.m}
\subsection{Wyniki}
\subsubsection{Transmitancja dyskretna}
$$
G(z)=\frac{0.05183 z^2 - 0.07375 z + 0.0259}{z^3 - 2.82 z^2 + 2.065 z - 0.4493}
$$
\subsubsection[Bieguny i zera]{Bieguny i zera transmitancji ciągłej i dyskretnej}
\begin{enumerate}
\item Transmitancja ciągła
\begin{itemize}
\item Bieguny: \verb+6 -8 -6+
\item Zera: \verb+-4.5 -2.5+
\end{itemize}

\item Transmitancja dyskretna
\begin{itemize}
\item Bieguny: \verb+1.8221 0.5488 0.4493+
\item Zera: \verb+ -4.5 -2.5+
\end{itemize}
\end{enumerate}

\begin{figure}[t]
\centering
\includegraphics[width=10cm]{../rys/rys1}
\caption{Zera i bieguny transmitancji ciągłej}
\label{fig:rys 1}
\end{figure}

\begin{figure}[H]
\centering
\includegraphics[width=10cm]{../rys/rys2}
\caption{Zera i bieguny transmitancji dyskretnej}
\label{fig:rys 2}
\end{figure}

\subsubsection{Wnioski}



\end{document}